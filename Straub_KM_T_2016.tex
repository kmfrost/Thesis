%
% PROJECT: <ETD> Electronic Thesis and Dissertation Initiative
%   TITLE: LaTeX report template for ETDs in LaTeX
%  AUTHOR: Kayla Straub, kstraub@vt.edu
% SAVE AS: Straub_KM_T_2016.tex
% REVISED: February 17, 2016
% 


\documentclass[12pt,dvips]{report}

\setlength{\textwidth}{6.5in}
\setlength{\textheight}{8.5in}
\setlength{\evensidemargin}{0in}
\setlength{\oddsidemargin}{0in}
\setlength{\topmargin}{0in}

\setlength{\parindent}{0pt}
\setlength{\parskip}{0.1in}

% Uncomment for double-spaced document.
\renewcommand{\baselinestretch}{2}

% \usepackage{epsf}

\begin{document}

\thispagestyle{empty}
\pagenumbering{roman}
\begin{center}

% TITLE
{\Large 
Data Mining Acadeic Emails to Model Employee Behaviors and Analyze Organizaitonal Structure
}

\vfill

Kayla M. Straub

\vfill

Thesis submitted to the Faculty of the \\
Virginia Polytechnic Institute and State University \\
in partial fulfillment of the requirements for the degree of

\vfill

Master of Science \\
in \\
Electrical Engineering
\vfill

Robert W. McGwier, Chair \\
Aloysius A. Beex\\
Michael Buehrer\\
Bert Huang

\vfill

March 25, 2016 \\
Blacksburg, Virginia

\vfill

Keywords: 
\\
Copyright 2016, Kayla M. Straub

\end{center}

\pagebreak

\thispagestyle{empty}
\begin{center}

{\large Data Mining Academic Emails to Model Employee Behaviors and Analyze Organizational Structure}

\vfill

Kayla M. Straub

\vfill

(ABSTRACT)

\vfill

\end{center}

Email correspondence has become the predominant method of communication for businesses.  
If not for the inherent privacy concerns, this electronically searchable data could be used to better understand how employees interact. 
For example, after the Enron dataset was made available, researchers were able to provide great insight into employee behaviors based on the available data despite the many challenges with that dataset.  
The work in this paper demonstrates the application of a suite of methods to an appropriately anonymized email dataset created from volunteers' email metadata.  
This new dataset, from an internal email server, is first used to validate machine learning and feature extraction algorithms and then to generate insight into the interactions within the center.  
Based solely on email metadata, a random forest approach modeled behavior patterns and accurately classified employees by job title.  
The algorithm performed very well not only on participants in the study but also on other members of the center who were connected to participants through email. 
Furthermore, the data revealed relationships not present in the formal operating structure. 
The result is a much fuller understanding of the center's internal structure than can be found in the official organization chart.


\vfill

% GRANT INFORMATION


\pagebreak

% Dedication and Acknowledgments are both optional
\chapter*{Dedication}
\chapter*{Acknowledgments}

\tableofcontents
\pagebreak

\listoffigures
\pagebreak

\listoftables
\pagebreak

\pagenumbering{arabic}
\pagestyle{myheadings}

\chapter{Introduction}
Build up context for the report.  Describe the problem and challenges associated with it.  Why is this important?  Brifely cover the process/solution I have implemented.  Touch on potential applications of this work.


\chapter{Literature Review}
Cover background works in similar studies, other email analyses [mostly Enron], and graph theory fundamentals.  Highlight how few studies have been performed before in this domain as well as their lack of quantifiable results.  Show the limitations of the Enron dataset.


\chapter{Data Collection}
Discuss the need for a modern, clean dataset.  Intimate knowledge of the environment that produced this data is to our advantage.  Describe the steps taken to collect, anonymize, and store the data properly.  Provide detailed descriptions/statistics of the dataset.

Subchapter: Enron issues
Subchapter: Collection process
Subchapter: Dataset statistics

\chapter{Feature Analysis}
Go into depth explaining all features used.  Cover the math behind all graph-based metrics.

Subchapter: Graph-based features
Subchapter: Social-based features
Subchapter: Feature selection

\chapter{Algorithm Design}
Discuss model selection and features selection.  Describe the design process and justify all choices made.

Subchapter: Learning algorithm

\chapter{Implementation}
Cover the process by which everything was calculated and implemented.  Talk about the software packages used and any special techniques.

\chapter{Testing and Results}
Talk about the testing process.  What did and did not work.  Describe the results and what they mean.

Subchapter: Proving behavior is constant in time
Subchapter: Classification results
Subchapter: Hierarchy analysis

\chapter{Conclusions}
Summarize the results.  Discuss the inferences that can be drawn from this result. 

\chapter{Future Work}
Discuss future work/what you could do with a larger dataset.
\\If I had more data I could cluster but it would be rough.  If I could label a few from each class, take 2 layers of neurons (greedily train).  Use Boltzman's and backprop to get an awesome classifier!  Write 3-4 pages on what more, higher-dimensional data + deep learning could improve.  Show that I know how it works.


\chapter{Applications}
Go into detail about what the applications are of this result.  Cover: corporate espionage and alternative communication networks (call records, text message metadata, social networks, etc.).


%%%%%%%%%%%%%%%%%
%
% Include an EPS figure with this command:
%   \epsffile{filename.eps}
%

%%%%%%%%%%%%%%%%
%
% Do tables like this:

% \begin{table}
% \caption{The Graduate School wants captions above the tables.}
%\begin{center}
% \begin{tabular}{ccc}
% x & 1 & 2 \\ \hline
% 1 & 1 & 2 \\
% 2 & 2 & 4 \\ \hline
% \end{tabular}
%\end{center}
% \end{table}

%%%%%%%%%%%%%%%%%%%%%%%%%%%%%%%%

% If you are using BibTeX, uncomment the following:
% \thebibliography
%
% Otherwise, uncomment the following:
\chapter*{Bibliography}

% \appendix

% In LaTeX, each appendix is a "chapter"
% \chapter{Program Source}


\end{document}

